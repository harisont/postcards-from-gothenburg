% Options for packages loaded elsewhere
\PassOptionsToPackage{unicode}{hyperref}
\PassOptionsToPackage{hyphens}{url}
%
\documentclass[
]{article}
\usepackage[swedish]{babel}
\usepackage{amsmath,amssymb}
\usepackage{lmodern}
\usepackage{iftex}
\ifPDFTeX
  \usepackage[T1]{fontenc}
  \usepackage[utf8]{inputenc}
  \usepackage{textcomp} % provide euro and other symbols
\else % if luatex or xetex
  \usepackage{unicode-math}
  \defaultfontfeatures{Scale=MatchLowercase}
  \defaultfontfeatures[\rmfamily]{Ligatures=TeX,Scale=1}
\fi
% Use upquote if available, for straight quotes in verbatim environments
\IfFileExists{upquote.sty}{\usepackage{upquote}}{}
\IfFileExists{microtype.sty}{% use microtype if available
  \usepackage[]{microtype}
  \UseMicrotypeSet[protrusion]{basicmath} % disable protrusion for tt fonts
}{}
\makeatletter
\@ifundefined{KOMAClassName}{% if non-KOMA class
  \IfFileExists{parskip.sty}{%
    \usepackage{parskip}
  }{% else
    \setlength{\parindent}{0pt}
    \setlength{\parskip}{6pt plus 2pt minus 1pt}}
}{% if KOMA class
  \KOMAoptions{parskip=half}}
\makeatother
\usepackage{xcolor}
\IfFileExists{xurl.sty}{\usepackage{xurl}}{} % add URL line breaks if available
\IfFileExists{bookmark.sty}{\usepackage{bookmark}}{\usepackage{hyperref}}
\hypersetup{
  hidelinks,
  pdfcreator={LaTeX via pandoc}}
\urlstyle{same} % disable monospaced font for URLs
\setlength{\emergencystretch}{3em} % prevent overfull lines
\providecommand{\tightlist}{%
  \setlength{\itemsep}{0pt}\setlength{\parskip}{0pt}}
\setcounter{secnumdepth}{-\maxdimen} % remove section numbering
\ifLuaTeX
  \usepackage{selnolig}  % disable illegal ligatures
\fi

\author{}
\date{}

\begin{document}

\hypertarget{att-beskriva-med-ljus-en-reflektion-uxf6ver-fotografi-och-realism}{%
\section{\texorpdfstring{\emph{Att (be)skriva med ljus}: en reflektion
över \\ fotografi och
realism}{Att (be)skriva med ljus: en reflektion över fotografi och realism}}\label{att-beskriva-med-ljus-en-reflektion-uxf6ver-fotografi-och-realism}}

\hypertarget{teman}{%
\subsection{Teman}\label{teman}}

\emph{Att skriva med ljus} \cite{asml} är en samling av 13 essäer av skandinaviska
författare. Även om titeln, som hänvisar till den etymologiska
betydelsen av ordet \emph{fotografi}, skulle passa lika bra för en
antologi med texter om fotografi i allmänhet, är skrivandet lika viktigt
som fotografin i den här boken.

Huvudtemat är nämligen förhållandet mellan fotografi och litteratur. Det
som gör boken särskilt intressant är dock den breda definition av
litteratur som används: olika författare analyserar olika typer av
fotolitterära verk, från romaner som Bretons \emph{Nadja} och
Gustafsons \emph{Kungsleden} till reportage (till exempel Canettis
\emph{Die Stimmen von Marrakesh} och de verk som nämns i \emph{Den
sociala fotobildboken}), genom alla gråzoner mellan dem. Man skulle då
kunna säga att ett mindre uttalat men lika centralt tema är förhållandet
mellan fotografi och verklighet.

\hypertarget{relation-till-kursprojektet}{%
\subsection{Relation till
kursprojektet}\label{relation-till-kursprojektet}}

Det är detta andra tema, med tanke på dess relevans för mitt projekt,
som jag vill fokusera på i denna uppsats.

I mitt projekt, som, med tillräckligt välgjorda bilder, skulle kunna
lika bra fungera helt utan text, är det själva fotografin som ibland
blir medvetet orealistiskt och nästan fiktivt. Tanken är att omvandla
arkitekturer som man ser i vardagen så att de ser annorlunda ut. I vissa
fall handlar det mest om estetik och abstraktion, men det finns också
fall där syftet är att omvandla till exempel den funktion som en byggnad
uppfattas ha: en fabrik kan se ut som en kyrka, en skyskrapa kan påminna
en rymdskepp, vattenfallet i ett vattenkraftverk kan framställas som om
det vore en del av ett naturreservat, och så vidare.

\hypertarget{reflektion-uxf6ver-utvalda-essuxe4er}{%
\subsection{Reflektion över utvalda
essäer}\label{reflektion-uxf6ver-utvalda-essuxe4er}}

I denna mening står den användning av fotografi som jag gör i mitt
projekt i skarp kontrast till vad de tidiga fotograferna och deras
samtida författare på 1800- och 1900-talet tilldelade fotografiet. Redan
i bokens inledningen citerar Mats Jansson Paul Valéry, som på
hundraårsdagen av av dagerrotypins födelse tänkte sig att ``spridningen
av fotografiska bilder {[}\ldots{]} kunde indirekt vara till gagn för
skönlitteraturen''\footnote{Gardfors, Jansson, Olsson, \textit{Att skriva med ljus}, s. 10}.

Enligt Janssons tolking verkade fotografin vara ett mycket bättre
verktyg för att beskriva verkligheten. Samtidigt öppnade daguerreotypens
existens nya möjligheter för författarna när det gäller abstrakt
tänkande och för poetiska uttrycksformer.

Tanken är att kameror gör det lätt att skapa så kallade ``realistiska'',
dokumentära bilder och att all arbete som kräver en hög grad av
objektivitet kan därför delegeras till fotografiet. På så sätt kan
författaren frigöra sig från beskrivande krav och koncentrera sig på mer
rent litterära aspekter.

På samma sätt, och kanske mer känt för allmänheten, förändrade
fotografiet måleriets värld. Om fotografiet gav till exempel
impressionisterna ett verktyg för att studera ljusets effekterna, ledde
det också till att måleriet, åtminstone i viss mån, avlägsnade sig från
realismen (i alla fall från ambitionen att uppnå realism) och förlorade
sin dokumentära funktion.

Skriver Gertrude Stein, citerad i Nils Olssons \emph{Painting looks like
something and photography does not}:

\begin{quote}
The painter can no longer say that what he does is as the world looks to
him because he cannot look at the world any more, it has been
photographed too much and he has to say that it does something else. In
former times a painter said he painted what he saw of course it didn't
but anyway he could say it, now he does not want to say it because
\emph{seeing it is not interesting}.\footnote{Gardfors, Jansson, Olsson, \textit{Att skriva med ljus}, s. 93}
\end{quote}

Kommenterar Nils Olsson att fotografiet ersätter inte måleriet, met det
gör måleriets begränsningar synliga. ``Målaren'', enligt Olsson, ``gör
något annat än fotografen, något annan än att förmedla hur världen ser
ut, vilket hädanefter saknar intresse för den med pensel''\footnote{\textit{Ibid.}, s. 93}.

Under sina första årtionden införde fotografiet således en ny
rollfördelning mellan de som arbetar med penseln och de nya
professionella med kameran och mörkrummet.

I \emph{Handsken i labyrintens mitt} skriver även Kristoffer Noheden
att surrealismens grundare André Bretons medieteorin är att fotografin
befriar måleriet från dess krav på mimetisk realism och tar dess
dokumentär funktion. Ett fotografi är ett mer effektivt sätt att fånga
verkligheten, så målaren öppnas helt nya möjligheter att utforska nya
uttrycksmedel.

Det var dock bara lite senare som fotografiet slutade vara helt enkelt
en ersättning för måleri på vissa områden och etablerade sig som en egen
konstform. Fortsätter Noheden:

\begin{quote}
Samtidigt har filmen utvecklat fotografin i nya riktningar och därmed
skänkt konsten ännu fler nya utmaningar. Filmen visar på att fotografin
är mer än ett fulländat mimetiskt medium - den är ett verktyg för att
först fånga in och sedan \emph{omvandla verkligheten}.\footnote{\textit{Ibid.}, s. 70}
\end{quote}

Breton och andra surrealistiska konstnärer brukade använda olika
omvandlingstekniker och ge ``helt nya perspektiv på vardagliga
händelser'', till exempel ``genom att sätta fotografier i rörelse och
introducera tekniker som slowmotion och baklängesrörelser, liksom
komprimerad eller utdragen tid genom klippning''\footnote{Gardfors, Jansson, Olsson, \textit{Att skriva med ljus}, s. 70}.

Även om Noheden (liksom de surrealister som hans text handlar om) ägnar
sig huvudsakligen åt rörliga bilder, finns det oändliga möjligheter att
omvandla och eventuellt förvränga även helt vanliga scener. Man kan till
exempel även välja en ovanlig synpunkt, visa endast en detalj, dra nytta
av ovanliga ljusförhållanden eller använda färg på ett kreativt sätt.
Detta utan att nämna redigeringen, som kan drivas till sin spets med
dagens digitala bildbehandling, men som på sätt och vis alltid har varit
en viktig del av fotografin även under den analoga eran, när den tog
formen av framkallning i mörkrummet. Då hade man möjligheten att vara
lika experimentell genom tekniker som \emph{film soup}.

Vi konstaterade att fotografin är \emph{mer än}, \emph{inte bara} ett
verktyg för att skapa en mimetisk realism. Men man kan till och med
ifrågasätta idén om att fotografi är ``ett fulländat mimetiskt medium''\footnote{\textit{Ibid.}, s. 70},
att fotografi är realistiskt.

Enligt John Berger, som nämns i essän \emph{Bildens vikt, poesins svikt} handlar fotografi om specifika sätt att se (\emph{ways of seeing}). I
det här fallet har ordet ``se'' inte samma betydelse som i Steins
skrifter. Stein skriver att att representera världen precis som man ser
den är ointressant, medan Bergers titel verkar antyda motsatsen: vad som
är intressant om ett fotografi är faktiskt fotografets blick. Det beror
på att Berger erkänner att det finns en mycket subjektiv komponent i
själva seendet, medan Stein anser att subjektiviteten ligger i den
efterföljande omarbetningen av det som uppfattas genom synen.

Oavsett hur ordet tolkas uppstår frågan om subjektivitet (och därmed
icke-objektivitet) är inneboende i varje fotografi. Paradoxalt verkar
det som om den ``kreativa omvandling'' av stadsarkitekturen som är
central i mitt projekt är definitivt möjlig, men att det inte är så
uppenbart att ``objektivera'' samma byggnader. Att imitera
``traditionella vykort'' är också ett särskilt filter, även om det gör
platserna mer igenkännliga.

På sätt och vis är fotografiets realism bara skenbar, och en subjektiv,
kanske till och med fiktiv, ibland skönlitterär komponent är alltid
närvarande.

\bibliographystyle{plain}
\bibliography{references.bib}

\end{document}
